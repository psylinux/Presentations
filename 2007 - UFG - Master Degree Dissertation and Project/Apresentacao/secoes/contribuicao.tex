\section{Nossa Contribui��o}

\begin{frame}
	\frametitle{Nossa Contribui��o}
  
		\begin{center}
				\LARGE Nossa principal contribui��o � o \textbf{Projeto Arquitetural de um Firewall Pessoal para Dispositivos M�veis}, que se utiliza do paradigma de localiza��o do usu�rio para carregar prefer�ncias e pol�ticas de seguran�a.
		\end{center}
		
\end{frame}


\begin{frame}[allowframebreaks]
	\frametitle{Nossa Contribui��o}
	  
				\begin{itemize}					
					\item \alert{\textbf{[SEGURAN�A]}} \\
					Aumentar o n�vel de seguran�a contra os ataques vindo da rede de comunica��o;
					\item \alert{\textbf{[CONTROLE]}} \\
					Controlar o acesso aos recursos do dispositivo m�vel e as informa��es;					
					\item \alert{\textbf{[POL�TICAS E PREFER�NCIAS]}} \\
					Flexibilidade para que o usu�rio execute, crie e edite suas prefer�ncias e pol�ticas de seguran�a;					
				\end{itemize}

\end{frame}	

	

\begin{frame}[allowframebreaks]
	\frametitle{Diferencial do F�nix Firewall System}

		\begin{itemize}
					\item \alert{\textbf{[LOCALIZA��O]}} \\
					Carregamento de Pol�ticas e Prefer�ncias baseado na localiza��o do usu�rio;
					\item \alert{\textbf{[INTERATIVIDADE CONTROLADA]}} \\
					N�vel de Interatividade controlada pelo usu�rio;
					\item \alert{\textbf{[META-POL�TICAS]}} \\
					Meta-pol�ticas export�veis;
					\item \alert{\textbf{[NOTIFICA��O]}} \\
					Servi�o de notifica��o do usu�rio via Web;					
		\end{itemize}

\end{frame}