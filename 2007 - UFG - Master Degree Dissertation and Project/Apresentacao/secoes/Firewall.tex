\chapter{Firewall}
Neste cap�tulo ser�o apresentados os conceitos b�sicos para a compreens�o do ambientes do F�nix Firewall System e dos servi�os oferecidos pelo mesmo. Como forma de contextualiza��o ser� apresentado uma breve descri��o do conceito de firewall, seus aspectos funcionais e import�ncia como uma camada de defesa para o usu�rio. Consequentemente ser� apresentado o conceito de firewall pessoal, que vai al�m do paradigma dos firewall tradicionais. Por fim ser�o apresentados alguns conceitos de computa��o ubiqua e contexto computacional encontrados na literatura.

\section{Sistemas de Firewall}


\section{Firewall Pessoal}


\section{Computa��o ub�qua e Contexto Computacional}
